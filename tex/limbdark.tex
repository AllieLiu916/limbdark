\documentclass[modern]{aastex61}

% All the packages
%\usepackage[letterpaper]{geometry}
\usepackage{fontspec}
\usepackage{microtype}
\usepackage{url}
\usepackage{amsmath}
\usepackage{mathtools}
\usepackage{esint}
\usepackage{amssymb}
\usepackage{natbib}
\usepackage{multirow}
\usepackage{graphicx}
\usepackage{scalerel}
\usepackage{calc}
\usepackage{etoolbox}
\usepackage{marginnote}
\usepackage{nicefrac}
\usepackage{tabstackengine}
\usepackage{diagbox}
\usepackage[makeroom]{cancel}
\usepackage{mathdots}
\usepackage{bbm}
\usepackage{booktabs}
\usepackage{xspace}
\usepackage{upgreek}
\usepackage[T1]{fontenc} % https://tex.stackexchange.com/a/166791
\usepackage{textcomp}
\usepackage{ifxetex}
\ifxetex
\usepackage{fontspec}
\defaultfontfeatures{Extension = .otf}
\fi
\usepackage{fontawesome}
\usepackage{listings}
\usepackage{mathtools}
\stackMath

% Page breaks in long equations
%\allowdisplaybreaks

% Bibliography stuff
\bibliographystyle{aasjournal}

% Shorthand for this paper
\newcommand{\starry}{\textsf{starry}\xspace}
\newcommand{\batman}{\textsf{batman}\xspace}
\newcommand{\planetplanet}{\textsf{planetplanet}\xspace}
\newcommand{\Python}{\textsf{Python}\xspace}
\newcommand{\cpp}{\textsf{C}++\xspace}
\newcommand{\Mathematica}{\textsf{Mathematica}\xspace}

% editing
\newcommand{\todo}[1]{{\color{red}\textbf{TODO:} #1}}


% References to text content
\newcommand{\documentname}{\textsl{article}}
\newcommand{\figureref}[1]{\ref{fig:#1}}
\newcommand{\Figure}[1]{Figure~\figureref{#1}}
\newcommand{\figurelabel}[1]{\label{fig:#1}}
\renewcommand{\eqref}[1]{\ref{eq:#1}}
\newcommand{\Eq}[1]{Equation~(\eqref{#1})}
\newcommand{\eq}[1]{\Eq{#1}}
\newcommand{\eqalt}[1]{Equation~\eqref{#1}}
\newcommand{\eqlabel}[1]{\label{eq:#1}}

% Add code, proof, and animation hyperlinks
\definecolor{linkcolor}{rgb}{0.1216,0.4667,0.7059}
\newcommand{\codeicon}{{\color{linkcolor}\faFileCodeO}}
\newcommand{\prooficon}{{\color{linkcolor}\faPencilSquareO}}
\newcommand{\animicon}{{\color{linkcolor}\faPlayCircle}}
\input{gitlinks}

\newtagform{eqtag}[]{(}{)}
\newcommand{\currentlabel}{None}

% Define a proof environment
\newenvironment{proof}[1]{%
\ifstrempty{#1}{%
\renewtagform{eqtag}[]{\raisebox{-0.1em}{{\color{red}\faPencilSquareO}}\,(}{)}%
}{%
\renewtagform{eqtag}[]{\prooflink{#1}\,(}{)}%
}%
\usetagform{eqtag}%
\renewcommand{\currentlabel}{#1}
\align%
}{%
\endalign%
\renewtagform{eqtag}[]{(}{)}%
\usetagform{eqtag}%
\message{<<<\currentlabel: \theequation>>>}%
}

% Define a proof environment
\newenvironment{proof*}[1]{%
\ifstrempty{#1}{%
\renewtagform{eqtag}[]{\raisebox{-0.1em}{{\color{red}\faPencilSquareO}}\,(}{)}%
}{%
\renewtagform{eqtag}[]{\prooflink{#1}\,(}{)}%
}%
\usetagform{eqtag}%
\renewcommand{\currentlabel}{#1}
\equation%
}{%
\endequation%
\renewtagform{eqtag}[]{(}{)}%
\usetagform{eqtag}%
\message{<<<\currentlabel: \theequation>>>}
}

% Math stuff
%\newcommand{\ii}{\ensuremath{\mathbf{i}}}
\newcommand{\T}{\ensuremath{\mathrm{T}}}
\newcommand{\dd}{\ensuremath{ \mathrm{d}}}
\newcommand{\unit}[1]{{\ensuremath{\mathrm{#1}}}}
\newcommand{\bvec}[1]{{\ensuremath{\mathbf{#1}}}}
\newcommand{\avec}[1]{{\ensuremath{\vec{\mathbf{#1}}}}}
\newcommand{\x}{\ensuremath{\mbox{$x$}}}
\newcommand{\y}{\ensuremath{\mbox{$y$}}}
\newcommand{\z}{\ensuremath{\mbox{$z$}}}
\newcommand{\xhat}{\ensuremath{\mathbf{\hat{x}}}}
\newcommand{\yhat}{\ensuremath{\mathbf{\hat{y}}}}
\newcommand{\zhat}{\ensuremath{\mathbf{\hat{z}}}}
\DeclareMathAlphabet\mathbfcal{OMS}{cmsy}{b}{n}
\DeclareMathOperator{\Tr}{Tr}
\DeclarePairedDelimiter\ceil{\lceil}{\rceil}
\DeclarePairedDelimiter\floor{\lfloor}{\rfloor}
\definecolor{dim}{rgb}{0.8,0.8,0.8}
\newcolumntype{L}[1]{>{\raggedright\let\newline\\\arraybackslash\hspace{0pt}}m{#1}}
\setcounter{MaxMatrixCols}{20}
\newcommand{\sinphi}{\ensuremath{\mbox{$u$}}}
\newcommand{\sinlambda}{\ensuremath{\mbox{$v$}}}
\newcommand{\bigdot}{\scaleto{\cdot}{6pt}}

% Bases
\newcommand{\ubasis}{\ensuremath{\tilde{\mathbf{u}}}}
\newcommand{\ubasisn}{\ensuremath{\tilde{\mathbf{u}}_n}}
\newcommand{\pbasis}{\ensuremath{\tilde{\mathfrak{p}}}}
\newcommand{\pbasisn}{\ensuremath{\tilde{\mathfrak{p}}_n}}
\newcommand{\gbasis}{\ensuremath{\tilde{\mathfrak{g}}}}
\newcommand{\gbasisn}{\ensuremath{\tilde{\mathfrak{g}}_n}}

% Code examples
\usepackage{listings}
\definecolor{codegreen}{rgb}{0,0.6,0}
\definecolor{codegray}{rgb}{0.5,0.5,0.5}
\definecolor{codepurple}{rgb}{0.58,0,0.82}
\definecolor{backcolour}{rgb}{0.95,0.95,0.95}
\lstdefinestyle{mystyle}{
    backgroundcolor=\color{backcolour},
    commentstyle=\color{codegreen},
    keywordstyle=\color{magenta},
    numberstyle=\tiny\color{codegray},
    stringstyle=\color{codepurple},
    basicstyle=\small\ttfamily,
    breakatwhitespace=false,
    breaklines=true,
    captionpos=b,
    keepspaces=true,
    numbers=left,
    numbersep=5pt,
    showspaces=false,
    showstringspaces=false,
    showtabs=false,
    tabsize=2,
    aboveskip=1em,
    belowskip=1em,
    keywords=[2]{map},
    keywordstyle=[2]{\color{black!80!black}},
}
\lstset{style=mystyle}

% Inverse diagonal dots
\makeatletter
\def\Ddots{\mathinner{\mkern1mu\raise\p@
\vbox{\kern7\p@\hbox{.}}\mkern2mu
\raise4\p@\hbox{.}\mkern2mu\raise7\p@\hbox{.}\mkern1mu}}
\makeatother

% Typography obsessions
\setlength{\parindent}{3.0ex}
\renewcommand\quad{\hskip\fontdimen3\font}


\begin{document}%\raggedbottom\sloppy\sloppypar\frenchspacing

\setlength{\abovedisplayskip}{1.5em}
\setlength{\belowdisplayskip}{1.5em}

\title{%
    Analytical Transit Light Curves for Limb-Darkened Stars
}

\author[0000-0002-0296-3826]{Rodrigo Luger}
\affil{Department~of~Astronomy, University~of~Washington, Seattle, WA}
\author{Eric Agol}
\affil{Department~of~Astronomy, University~of~Washington, Seattle, WA}

\keywords{methods: analytical --- techniques: photometric}

\begin{abstract} 
    We derive analytical, closed form solutions for the light curve
    of a planet transiting a star with a limb darkening profile of
    arbitrary order. We provide updated expressions for the linear
    and quadratically limb darkened cases that are numerically stable
    over the entire domain. 
\end{abstract}

% ==============================================================================
% ------------------------------------------------------------------------------
% ------------------------------------------------------------------------------
%
\section{Introduction}
\label{sec:intro}
% ------------------------------------------------------------------------------
% ------------------------------------------------------------------------------
% ==============================================================================

Currently I just copy-pasted some relevant stuff from the \starry paper.
This week I'll organize this into sections and get an outline going.\todo{}


% ==============================================================================
% ------------------------------------------------------------------------------
% ------------------------------------------------------------------------------
%
\section{Reparametrization}
\label{sec:reparam}
% ------------------------------------------------------------------------------
% ------------------------------------------------------------------------------
% ==============================================================================

From \citet{MandelAgol2002}, the total flux visible during the occultation of a
body whose surface map is given by $I(x, y) = \sqrt{1 - \x^2 -\y^2}$ may be computed
as
%
\begin{align}
    \label{eq:s2}
    s_2 = \frac{2\pi}{3} \left(1 - \frac{3\Lambda}{2} - \Theta(r - b) \right)
\end{align}
%
where $\Theta(\bigdot)$ is the Heaviside step function and
%
\begingroup\makeatletter\def\f@size{10}\check@mathfonts
\def\maketag@@@#1{\hbox{\m@th\large\normalfont#1}}%
\begin{align}
    \label{eq:biglam}
    \Lambda &=
    \begin{dcases}
          % I don't think we need this: the k^2>1 term is stable as b --> 0!
          %-\frac{2}{3}\left(1 - r^2\right)^\frac{3}{2}
          %& \qquad b = 0
          %
          %\\[1.5em]
          %
          \frac{1}{9 \pi \sqrt{b r}} \Bigg[
                \frac{(r + b)^2 - 1}{r + b}
                \Big(
                    -2r \,
                    \big(
                        2 (r + b)^2 + (r + b)(r - b) - 3
                    \big)
                    K(k^2)
                    &\\ \phantom{XXXX}
                    + 3 (b - r) \, \Pi\big(k^2 (b + r)^2, \, k^2\big)
                \Big)
                - 4 b r (4 - 7 r^2 - b^2) E(k^2)
          \Bigg]
          %
          & \qquad k^2 < 1
          %
          \\[1.5em]
          %
          \frac{2}{9 \pi} \Bigg[
                \big(1 - (r + b)^2\big)
                \Bigg(
                    \sqrt{1 - (b - r)^2} \,
                    K\left(\frac{1}{k^2}\right)
                    + 3 \left(\frac{b-r}{(b+r)\sqrt{1 - (b - r)^2}}\right)
                    &\\ \phantom{XX}
                    \times \Pi\left(\frac{1}{k^2(b+r)^2}, \, \frac{1}{k^2}\right)
                \Bigg)
                - \sqrt{1 - (b - r)^2}
                (4 - 7 r^2 - b^2)
                E\left(\frac{1}{k^2}\right)
          \Bigg]
          %
          & \qquad k^2 \ge 1
    \end{dcases}
\end{align}
\endgroup
%
with
%
\begin{align}
    \label{eq:k2}
    k^2 &= \frac{1 - r^2 - b^2 + 2 b r}{4 b r}
    \quad.
\end{align}
%
In the expressions above, $K(\bigdot)$, $E(\bigdot)$, and $\Pi(\bigdot, \bigdot)$
are the complete elliptic integrals of the first, second kind, and third kind,
respectively, defined as
%
\begin{align}
    \label{eq:elliptic}
    K(k^2) &\equiv \int_0^{\frac{\pi}{2}} \frac{\dd \varphi}{\sqrt{1 - k^2 \sin^2 \varphi}}
    \nonumber \\[0.5em]
    E(k^2) &\equiv \int_0^{\frac{\pi}{2}} \sqrt{1 - k^2 \sin^2 \varphi} \, \dd \varphi
    \nonumber \\[0.5em]
    \Pi(n, k^2) &\equiv \int_0^{\frac{\pi}{2}} \frac{\dd \varphi}{(1 - n \sin^2 \varphi)\sqrt{1 - k^2 \sin^2 \varphi}}
    \quad.
\end{align}

The solution for $s_2$ (Equation~\ref{eq:s2}) becomes unstable as $b \rightarrow r$ because
the elliptic integral $\Pi$ diverges. In the vicinity of
$b = r$ we use Equation (17.7.14) in \citet{Abramowitz1970} to express $\Pi(n, k^2)$
in terms of Heuman's Lambda function.

The $s_2$ term is also unstable when $r$ (and $b$) become much greater than unity, since
in this limit $k^2 \rightarrow 0$ and $E(k^2) \rightarrow K(k^2) \rightarrow \frac{\pi}{2}.$
Since $s_2$ depends on (among other things) a function of the difference between these two
elliptic integrals, roundoff error in their computation leads to catastrophic cancellation
in the result. In order to circumvent this, we re-write \eq{biglam} in terms of $E(k^2) - K(k^2)$
and Taylor expand the expression when $r > 1$ to high order in $k^2$.

% ==============================================================================
% ------------------------------------------------------------------------------
% ------------------------------------------------------------------------------
%
\section{Quadratic Limb-Darkening}
\label{sec:quad}
% ------------------------------------------------------------------------------
% ------------------------------------------------------------------------------
% ==============================================================================

Any radially symmetric specific intensity profile can be expressed as a sum
over the $m = 0$ spherical harmonics.
%
In particular, the radial intensity profile of a quadratically
limb-darkened star,
%
\begin{align}
    \label{eq:quadraticld}
    I(\upmu) &= 1 - u_1 (1 - \upmu) - u_2 (1 - \upmu)^2
    \quad,
\end{align}
%
where $\upmu = \z = \sqrt{1 - \x^2 - \y^2}$ and $u_1$ and $u_2$ are the
limb darkening coefficients, can be written in terms of spherical harmonics
by re-writing \eq{quadraticld} as
%
\begin{align}
    I(\x, \y) = (1 - u_1 - 2u_2) + (u_1 + 2u_2) \z + u_2 \x^2 + u_2 \y^2
           \quad.
\end{align}
%
Performing the change of basis to spherical harmonics \citep{starry}, we have
%
\begin{align}
    \label{eq:qldylm}
    I(\x, \y) =
            \frac{2\sqrt{\pi}}{3} (3 - 3u_1 + 4u_2) \, Y_{0,0}
          + \frac{2\sqrt{\pi}}{\sqrt{3}} (u_1 + 2u_2) \, Y_{1,0}
          - \frac{4\sqrt{\pi}}{3\sqrt{5}} u_2 \, Y_{2,0}
      \quad.
\mathematica{limbdark}
\end{align}
%
Thus, quadratic limb darkening can be expressed exactly as the sum of the first
three $m = 0$ spherical harmonics.

\bibliography{limbdark}

\end{document}
